\section*{ВВЕДЕНИЕ}
\addcontentsline{toc}{section}{ВВЕДЕНИЕ}


В современном мире информационных технологий веб-мессенджеры стали неотъемлемой частью нашей повседневной жизни. За последние десятилетия создание мессенджеров стало одним из самых востребованных направлений в программировании. Искусство разработки мессенджеров предоставляет уникальную возможность соединить техническое мастерство с креативностью и фантазией. В рамках данного курсового проекта мы погрузимся в увлекательный мир разработки веб-мессенджера и рассмотрим процесс создания современного приложения для обмена сообщениями с использованием технологий веб-разработки.

Наш веб-мессенджер предназначен для обеспечения эффективного и удобного общения пользователей в режиме реального времени, а также привлечения новых пользователей своей функциональностью и привлекательным дизайном.

В процессе разработки было создано веб-приложение, использующее современные веб-технологии, включая JavaScript, HTML и CSS. Реализованы основные элементы мессенджера, такие как чат, отправка сообщений, система регистрации и входа в систему.

При разработке использовались современные технологии веб-разработки, а также библиотеки и фреймворки, обеспечивающие удобный пользовательский интерфейс и безопасное взаимодействие.


\emph{Цель настоящей работы} –  создание современного веб-мессенджера с использованием технологий веб-разработки. Для достижения этой цели необходимо решить \emph{следующие задачи:}
\begin{itemize}
	\item провести анализ современных веб-мессенджеров;
	\item разработать концепцию и функциональные требования к мессенджеру;
	\item спроектировать веб-приложение с учетом требований безопасности и удобства использования;
	\item реализовать функционал мессенджера.
\end{itemize}

\emph{Структура и объем проекта.} Проект включает в себя введение, 4 раздела основной части, заключение, список использованных технологий, 2 приложения. Общий объем проекта равен \formbytotal{page}{страниц}{е}{ам}{ам}.

\emph{Во введении} сформулирована цель проекта, поставлены задачи разработки, описана структура проекта, приведено краткое содержание каждого из разделов.

\emph{В первом разделе} производится анализ существующих веб-мессенджеров и их технических характеристик.

\emph{Во втором разделе} формулируются требования к создаваемому веб-мессенджеру на этапе технического задания.

\emph{В третьем разделе} представлен детальный технический проект, включая выбор используемых технологий, архитектуру приложения и основные функциональные блоки.

\emph{В четвертом разделе} представлен исходный код разработанного мессенджера, проведено тестирование и оценка его производительности.

В заключении изложены основные результаты проекта, полученные в ходе разработки.

В приложении А представлен исходный код.
 
