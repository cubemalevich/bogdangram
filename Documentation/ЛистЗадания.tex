\newpage
\begin{center}
	\large\textbf{Минобрнауки России}
	
	\large\textbf{Юго-Западный государственный университет}
	\vskip 1em
	\normalsize{Кафедра программной инженерии}
	\vskip 1em
	\ifВКР{
		\begin{flushright}
			\begin{tabular}{p{.4\textwidth}}
				\centrow УТВЕРЖДАЮ: \\
				\centrow Заведующий кафедрой \\
				\hrulefill \\
				\setarstrut{\footnotesize}
				\centrow\footnotesize{(подпись, инициалы, фамилия)}\\
				\restorearstrut
				«\underline{\hspace{1cm}}»
				\underline{\hspace{3cm}}
				20\underline{\hspace{1cm}} г.\\
			\end{tabular}
		\end{flushright}
	}\fi
\end{center}
\vspace{1em}
\begin{center}
	\large
	\ifВКР{
		ЗАДАНИЕ НА ВЫПУСКНУЮ КВАЛИФИКАЦИОННУЮ РАБОТУ
		ПО ПРОГРАММЕ БАКАЛАВРИАТА}
	\else
	ЗАДАНИЕ НА КУРСОВУЮ РАБОТУ (ПРОЕКТ)
	\fi
	\normalsize
\end{center}
\vspace{1em}
{\parindent0pt
	Студента \АвторРод, шифр\ \Шифр, группа \Группа
	
	1. Тема «\Тема\ \ТемаВтораяСтрока»
	\ifВКР{
		утверждена приказом ректора ЮЗГУ от \ДатаПриказа\ № \НомерПриказа
	}\fi.
	
	2. Срок предоставления работы к защите \СрокПредоставления
	
	3. Исходные данные для создания программной системы:
	
	3.1. Перечень решаемых задач:}

\renewcommand\labelenumi{\theenumi)}

\begin{enumerate}
	\item ознакомиться с обработкой веб запросов;
	\item разработать серверную часть приложения;
	\item разработать клиентскую часть приложения;
	\item Доработать мессенджер и провести тестирование.
\end{enumerate}

{\parindent0pt
	3.2. Входные данные и требуемые результаты для программы:}

\begin{enumerate}
	\item Входными данными у мессенджера является возможность писать а так же принимать сообщения так же необходима система регистрации и входа, дабы отличать пользователей по их именам или же никнеймам. 
%	Выходными данными являются графическая интерпретация игрового процесса на мониторе пользователя. Действия игрока влияют на игровой про-цесс и текущее состояние игровой сцены. Игрок контролирует автомобиль с помощью интерфейса пользователя.
	
	\item  выходными данными являтеся тот факт что пользователь взаимодействует с приложением, предоставляя данные для регистрации, входа и отправки сообщений, и ожидает соответствующих результатов в виде успешных операций.
\end{enumerate}

{\parindent0pt
	
	4. Содержание работы (по разделам):
	
	4.1. Введение
	
	4.2. Анализ предметной области
	
	4.3. Техническое задание: основание для разработки, назначение разработки,
	требования к программной системе, требования к оформлению документации.
	
	4.4. Технический проект: общие сведения о программной системе, проект
	данных программной системы, проектирование архитектуры программной системы, проектирование пользовательского интерфейса программной системы.
	
	4.5. Рабочий проект: спецификация компонентов и классов программной системы, тестирование программной системы, сборка компонентов программной системы.
	
	4.6. Заключение
	
	4.7. Список использованных источников
	

	
	\списокПлакатов
	
	\vskip 2em
	\begin{tabular}{p{6.8cm}C{3.8cm}C{4.8cm}}
		Руководитель \ifВКР{ВКР}\else работы (проекта) \fi & \lhrulefill{\fill} & \fillcenter\Руководитель\\
		\setarstrut{\footnotesize}
		& \footnotesize{(подпись, дата)} & \footnotesize{(инициалы, фамилия)}\\
		\restorearstrut
		Задание принял к исполнению & \lhrulefill{\fill} & \fillcenter\Автор\\
		\setarstrut{\footnotesize}
		& \footnotesize{(подпись, дата)} & \footnotesize{(инициалы, фамилия)}\\
		\restorearstrut
	\end{tabular}
}

\renewcommand\labelenumi{\theenumi.}
